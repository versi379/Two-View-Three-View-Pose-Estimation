\section{Pose Estimation}\label{sec:estimation}
We can derive the epipoles, which are the projections of the first camera centre onto the second and third images, from a \acs{TFT} \( \mathbfcal{T} \). The epipole \( e_{21} \) is found as the common point of intersection among the lines represented by the left null-vectors of \( \bm{T}_1 \), \( \bm{T}_2 \), and \( \bm{T}_3 \). Similarly, the epipole \( e_{31} \) is determined as the shared point of intersection among the lines represented by the right null-vectors of \( \bm{T}_1 \), \( \bm{T}_2 \), and \( \bm{T}_3 \). Subsequently, we can compute the \acs{FM}s as
\begin{equation}
	\begin{gathered}
		F_{21} = [e_{21}]_{\times}[\bm{T}_1e_{31}, \bm{T}_2e_{31}, \bm{T}_3e_{31}]\\
		F_{31} = [e_{31}]_{\times}[\bm{T}_1^\top e_{21}, \bm{T}_2^\top e_{21}, \bm{T}_3^\top e_{21}].
	\end{gathered}
	\label{eq:fmFromEpiTFT}
\end{equation}

The \ac{EM} is the specialisation of the \acs{FM} to the case of normalized image coordinates. The \acs{EM}s here can be derived from \( F_{21}, F_{31} \), and the calibration matrices \( K_i \), using the formula \( [t_{ij}]_{\times}R_{ij} = E_{ij} = K_i^\top F_{ij}K_j \). From these \acs{EM}s, the relative orientations \( (R_{21}, t_{21}) \) and \( (R_{31}, t_{31}) \) can be extracted through the \acs{SVD} of \( E_{21} \) and \( E_{31} \), with each translation vector's scale remaining unknown. To establish an overall scale, we set \( \Vert t_{21} \Vert = 1 \), while the relative scale \( \lambda \) of \( t_{31} \) can be determined by triangulating the space points \( \{X^n\}_n \) from the first two cameras' projections and minimizing the algebraic error relative to the third image, as shown
\begin{equation}
	\argmin_{\lambda \in \mathbb{R}}{\sum_{n = 1}^{N}{\left\Vert x_3^n \times \left( K_3 \left( R_{31}X^n + \lambda \frac{t_{31}}{\Vert t_{31} \Vert} \right) \right) \right\Vert}}.
\end{equation}

The latter admits a closed form solution. So, either from the \acs{TFT} or the \acs{FM}s, we possess a method for computing the camera poses.

\subsection{Bundle Adjustment}
In pose estimation, a frequent final stage involves refining the orientations through \ac{BA}. This process aims to minimize the square reprojection error across potential camera orientations and spatial points. For N correspondences and M = 3 cameras
\begin{equation}
	\min_{\{ R_j, t_j \}_j, \{ X^i \}_i}{\epsilon^2}, \quad \epsilon^2 = \sum_{i = 1}^{N}{\sum_{j = 1}^{M}{d \left( x_j^i, K_j(R_jX^i + t_j) \right)^2}},
	\label{eq:baMin}
\end{equation}

where \( x_j^i \) is for the homogeneous coordinates of the observed image point, and the distance \( d \) is the Euclidean distance in homogeneous coordinates
\begin{equation}
	d \left( (x, y, z)^\top, (t, u, v)^\top \right)^2 = \left( \frac{x}{z} - \frac{t}{v} \right)^2 + \left( \frac{y}{z} - \frac{u}{v} \right)^2.
\end{equation}

The optimization procedure can be executed using the \acs{LM} algorithm \cite{14-levenberg}.

\begin{algorithm}[h]
		\caption{Pose Estimation Algorithm}
		\kwObjective{Given \acs{FM} or \acs{TFT}, extract camera poses.}
		\kwAlgorithm{
		\begin{enumerate}[label=(\roman*),leftmargin=*,rightmargin=1.5em]
        	\item If employing \acs{TFT}, derive the epipoles \( e_{21}, e_{31} \) first, and then compute the fundamental matrices \( F_{21}, F_{31} \) as stated in Equation (\ref{eq:fmFromEpiTFT}); otherwise (employing \acs{FM}) go to step (ii).
        	\item Compute essential matrices \( E_{21}, E_{31} \) from the fundamental matrices \( F_{21}, F_{31} \) and the calibration matrices \( K_i \).
        	\item Exploit essential matrices to determine rotations \( R_2, R_3 \) and translations \( t_2, t_3 \).
        	\item Apply \acs{BA} to refine rotations and translations (\ie, orientations), by minimizing the squared reprojection error as stated in Equation (\ref{eq:baMin}).
        	\item Triangulate 3D points from their image projections using the \ac{DLT} algorithm, obtaining the reconstructed 3D scene.
        \end{enumerate}
    }
\end{algorithm}
